\section* {Motivation letter}

I hereby apply for the position Besch\"{a}ftigte/r mit abgeschlossener wissenschaftlicher Hochschulbildung" at the Technische Universit\"{a}t Berlin - Universit\"{a}tsbibliothek
\vspace {0.5cm} 
  % \emph{A crapy (software) tool that is used is better than a perfect one without user.}

Dear Ms. Schobert,

I am a former neuro-biologist whose interest in the optimisation of the scientific process has brought to promote open science, open access and open data. I have worked as an enthusiastic and pragmatic science manager, and advocates that efficacy and efficiency goes hand in hand with being more open. During the last year, I got to learn more about the role of libraries and corporate industries in science communication and open science development. After presenting in detail my interest and previous work related to the three main activities of the job position, I will summarise why I think I am fitted for the position.





\subsection*{Talking Open Access to scientists}

Getting the attention of scientists while talking about anything different than a scientific topic they relate to is a very difficult task. One needs to be creative in both the channels used and the message to carry. Finding the right target audience is key, and one needs to get a quite large support before the community may change their habit. The process is very slow, as expressed by founders of Mendeley and Protocols.io. There is three routes I tried so far, and many more one could explore. Top down influence from the politics and the funders has recently be motivating scientists to get informed about open access, and we have more opportunities to get the message through.

\subsubsection*{Being an example and touch my own scientific community}
I published all my papers since 2014 in gold open access journals, checked that previous work was available via the green open access route and published all data and code with my papers under adequate licenses. I gave two presentations including open science principles in two Drosophila neurobiology specific conference in 2004 and 2006, where I was selected for oral presentations twice [2] and keep at least one open science slide in all my scientific presentation.
\subsubsection*{Create grassroots movement aimed at young researchers}
I pushed the development of open science communities (open science meetups series[1], mozilla science lab), networking in the startup scene for (open) science (labfolder, digital science, knowledge unlatched) and learning about the activities at the institutional level (OpenAIRE, FOSTER, FORCE11, TIB). 
\subsubsection*{Advocate for open science in courses for data management}
More recently, I have been developing teaching material to cover open science [3] and open data principles and practice[4]. I do thing that the different aspects of open science are feeding each other and that a complete approach will be more efficient at getting supporters. 



\subsection*{???}


\subsection*{TU Bibliography}

My basic knowledge in database development and the development of web applications using open databases [6], as well as my interest in corporate solution like ORCID, put me in a good position to organise the implementation of an existing ORCID based bibliography tool (\url{https://github.com/UB-Dortmund/mms}).

\rubrique {references}

\begin{itemize}[noitemsep]

\item [1] \url{https://www.meetup.com/Berlin-Open-Science-Meetup/}
\item [2]  \url{https://www.slideshare.net/JulienColomb/p-02-2014})
\item [3] \url{opensciencemooc.eu}
\item [4] \url{http://www.rpubs.com/j_colomb/256819}
\item [5] see for example \url{https://doi.org/10.6084/m9.figshare.5658895.v1}
\item [6] \url{https://colomb.shinyapps.io/Flystockcleaner/}

\end{itemize}
%\newpage 
\subsection*{Summary}
Owing to my education, my skill set, my goal-oriented and practical thinking, as well as my zeal, I believe I am qualified for the position and can help you reach the OA objectives in time. In particular, I have the scientific experience and the creativity to design new communication channels between libraries and scientists and I will be eager to create bridges between the library and the scientists and engineers at the TU.

 

I am  a quick learner, stress resistant and I like to work on ambitious projects lead by small teams. I also have a background in software development and collaborative working and team management which may come handy. My current contract is a 50\% position at the HU, it ends in March 2018 and I have no other remunerated activity at the moment. 
Since I left the lab in 2014, I have been trying to find new paths for my career and I have been involved in many different projects in parallel. I have learned to manage my time and the team work more effectively and I am currently looking for a financially stable way to work on open science in a full time basis. 

In brief, I feel like if we would work together, we could make a difference in the open access and open science landscape, in Berlin and beyond,
and I am looking forward to meet you in person.


Dr. Julien Colomb



\vspace {1.5cm} 

Referees:

Prof. York Winter, Berlin, york.winter@charite.de
 
 
Prof. Bjoern Brembs, Regensburg, bjoern@brembs.net
