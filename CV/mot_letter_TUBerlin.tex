\section* {Motivation letter}

I hereby apply for the position Besch\"{a}ftigte/r mit abgeschlossener wissenschaftlicher Hochschulbildung" at the Technische Universit\"{a}t Berlin - Universit\"{a}tsbibliothek
\vspace {0.5cm} 
  % \emph{A crapy (software) tool that is used is better than a perfect one without user.}

Dear Ms. Schobert,

I am a former neuro-biologist whose interest in the optimisation of the scientific process has brought to promote open science, open access and open data. I have been working as an enthusiastic and pragmatic science manager, and advocating that efficacy and efficiency goes hand in hand with being more open. During the last year, I got to learn more about the role of libraries and corporate industries in science communication and open science development. A detailed presentation of 
 my interest and previous work related to the main activities of the job position follows this first introduction. My CV was inserted afterwards.


Owing to my education, my skill set, my goal-oriented and practical thinking, as well as my zeal, I believe I am qualified for the position and can help you reach your objectives in time. In particular, I have the scientific experience and the creativity to design new communication channels between libraries and scientists and I will be eager to create bridges between the library and the scientists and engineers at the TU.

 

I am  a quick learner, stress resistant and I like to work on ambitious projects lead by small teams. I also have a background in software development, startup creation, collaborative working and team management which may be supplementary assets. My current contract is a 50\% position at the HU, it ends in March 2018 and I have no other remunerated activity at the moment. 
Since I left the lab in 2014, I have been trying to find new paths for my career and I have been involved in many different projects in parallel. I have learned to manage my time and the team work more effectively and I am currently looking for a financially stable way to work on open science on a full time basis. 

In brief, I feel like if we would work together, we could make a difference in the open access and open science landscape, in Berlin and beyond,
and I am looking forward to meet you in person and feel free to contact me for further information.


Dr. Julien Colomb



\vspace {1.cm} 

Referees:

Prof. York Winter, Berlin, york.winter@charite.de
 
 
Prof. Bjoern Brembs, Regensburg, bjoern@brembs.net
\newpage

\begin{multicols}{2}

\section{Talking Open Access and ORCID to scientists}

Getting the attention of scientists while talking about anything different than their scientific topic is a very difficult task. One needs to be creative in both the marketing channels used and the message to carry. Finding the right target audience is key, and one should expect a slow process.
%
There is three routes I tried so far to advocate for open science, and many more one could explore. I do think that the different aspects of open science are feeding each other and the same strategies will apply for the advocacy of open access and ORCID. 
\subsubsection*{Being an example and touch my own scientific community}
I published all my papers since 2014 in gold open access journals, checked that previous work was available via the green open access route and published all data and code with my papers under adequate licenses. I was an early adopter of ORCID and my ORCID data is up to date. I gave two presentations including open science principles in two Drosophila neurobiology specific conference in 2004 and 2006, where I was selected for oral presentations twice [2] and have at least one open science slide in all my scientific presentations since the last FORCE meeting.
\subsubsection*{Work on different levels: grassroot, startups and institutions}
I pushed the development of open science communities (open science meetups series[1], mozilla science lab), networking in the startup scene for (open) science (labfolder, digital science, knowledge unlatched) and learning about the activities at the institutional level (OpenAIRE, FOSTER, FORCE11, TIB). 
\subsubsection*{Advocate for open science in courses for data management}
More recently, I have been developing teaching material to cover open science [3] and open data principles and practice[4]. The different aspects of open science are feeding each other and an overall presentation has chances to be more efficient at getting supporters.  One strategy I would 
favor is to convince scientists they should do open science in the future and come right afterwards with indication about how the library can help them.

\subsection*{Open Access particularities}
Top down influence from the politics and the funders has recently be pushing scientists to get informed about open access, and we have more opportunities to get the message through. The strategy involving personal coaching and group workshop which was shown to be effective in data management initiatives at the EPFL will probably be effective for open access too.

\subsection*{ORCIDs particularities}
Scientists may see ORCID as a tool librarians want for their own purpose, and not as a tool for their own work and career and it will therefore be particularly difficult to persuade them that they should get and update an ORCID ID. The different advantages of ORCID for the scientists' carreer should be emphased [6]; in particular I would tone the fact that one can use the ID credential to sign up to very different platforms (zenodo, overleaf,science open,..) and populate different accounts with your data automatically.



\section{TU Bibliography}

My knowledge in programming, database development and the development of web applications using open databases [7], as well as my interest in corporate solution like ORCID, put me in a good position to organise the implementation of an ORCID based bibliography tool. Here I believe one should reach to other libraries to work collaboratively to create new tools or develop existing ones further (\url{https://github.com/UB-Dortmund/mms}).

\rubrique {references}

\begin{itemize}[noitemsep]

\item [1] \url{https://www.meetup.com/Berlin-Open-Science-Meetup/}
\item [2]  \url{https://www.slideshare.net/JulienColomb/p-02-2014})
\item [3] \url{https://opensciencemooc.eu/}
\item [4] \url{http://www.rpubs.com/j_colomb/256819}
\item [5] see for example \url{https://doi.org/10.6084/m9.figshare.5658895.v1}
\item [6]\url{ https://orcid.org/help}
\item [7] \url{https://colomb.shinyapps.io/Flystockcleaner/}


\end{itemize}
\end{multicols}
\newpage 

