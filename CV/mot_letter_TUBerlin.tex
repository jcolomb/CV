\section* {Motivation letter}

I hereby apply for the position Besch\"{a}ftigte/r mit abgeschlossener wissenschaftlicher Hochschulbildung" at the Technische Universit\"{a}t Berlin - Universit\"{a}tsbibliothek
\vspace {0.5cm} 
  % \emph{A crapy (software) tool that is used is better than a perfect one without user.}

Dear Ms. Schobert,

I am a scientist, my background is neurogenetics. During the last years, I have been relatively successful on advocating open science practice, especially via creating an open science meetup in Berlin [1] and presenting open science principles during scientific conferences [2]). While it all starts with Open Access, I have been more active in talking about open data and reproducible research, and tried to be an example in my own scientific publications. I believe that the three stakeholders (scientific community, university libraries, private companies) have their role in the transition toward open science, and I am always pleased to see the efforts and success of German libraries in making Open access a reality, especially the green way.

While reading the e-science tage 2017 report, I could see two areas where the libraries have problems. The first one is a lack of communication and open practice inside and between libraries. We see different teams reinventing the wheel and develping different software achieving similar goals. In particular, XXXXX report on their strategy and software to create university libraries with Orcid integration looks like a system to be copied. The second problem is the quasi absence of communication between scientists and libraries. At the last FORCE meeting, I could see nice discussion between libraries and corporate entities, but scientists were more or less excluded. It is a difficult task, and I think one needs to be creative, and copy the workflow of startups in the development of new tools and marketing strategies (test the product before its creation, agile development, 

If you are looking for a creative scientist with IT knowledge, someone with a very strong background in (open) science, reproducible research and critical thinking, someone who is passionate about open FAIR data, someone who can understand the need of researchers, depict how an IT solution can help them perform their tasks efficiently and effectively, we should definitively meet.

I am an experienced neuro-geneticist with a decade of experience in complex behavioural data production and analysis in fundamental research, a field where one learns quickly that data quality and experiment speed are correlated positively. I am the first author of the first ever published paper with a living figure \url{https://f1000research.com/articles/3-176/v2}, and I have published all the data and code of my papers since 2009 (under CC0 licenses, mostly on Figshare, publications appeared of course in gold open access journals). My latest project is an open science project where I am developing a repository for home cage monitoring data (and its analysis using machine learning algorithms). I am using github, osf and zenodo as data storage and R (and shiny) as an interface to access and analyse the data \url{github.com/jcolomb/HCS_analysis}. Independently, I have been involved in project aiming at teaching data management and open science practices (\url{opensciencemooc.eu}, \url{https://www.youtube.com/watch?v=M-Zod8o7nTg}, \url{http://access2perspectives.com/julien-colomb/}). I was also very interested in ontologies, open linked data and semantic web solutions, while I did not had the possibility to participate in this effort directly, yet.

I am one of the few who have both a very strong background in life science research, a decent background in creating IT solutions to help the scientific process and experience in team and project management (I could put the concepts learned in the "Moderation \& management" course during my time directing the animal outcome core facility). 
Independently, I have been running Drososhare GmbH and its IT solution development. (Drososhare is a web-platform aiming at facilitating peer to peer transactions of GMOs in the life sciences and make material descriptions of better quality in the literature.) During the last four years, I learned the difficulties in communicating one's wish to IT developers and how to be more effective at it. I got involved and interested in the Berlin startup communities, learning about agile development, business models and specificities of the scientists as customers. In addition, I am co-leading the open science berlin meetup group and was part of the march for science Berlin 2017 organisation team. I am also the happy father of two lovely children (4 and 1.5 years).

\newpage 

I am  a quick learner, stress resistant and I like to work on ambitious projects lead by small teams. My current contract is a 50\% position at the HU, it ends in March 2018 and I have no other remunerated activity at the moment. 
Since I left the lab in 2014, I have been trying to find new paths for my career and,
as you see, I have been involved in many different projects in parallel. I have learned to manage my time and the team work more effectively and I am currently looking for a financially stable way to focus my energy on one single project. 

In brief, I feel like if we would work together, we could make a difference in the open science landscape, in Hamburg and beyond,
and I am looking forward to meet you in person.

Please reach to me for additional information.

Dr. Julien Colomb

[1] https://www.meetup.com/Berlin-Open-Science-Meetup/
[2]


\vspace {1.5cm} 

Referees:

Prof. York Winter, Berlin, york.winter@charite.de
 
 
Prof. Bjoern Brembs, Regensburg, bjoern@brembs.net
