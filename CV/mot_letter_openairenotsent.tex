\section* {Motivation letter}

I hereby apply for the position Project Manager for the European OpenAIRE-Advance project'' at Open-AIRE
\vspace {0.5cm} 
  % \emph{A crapy (software) tool that is used is better than a perfect one without user.}

Dear Open-AIRE team,

Owing to my education, my skill set and my interest for the dissemination of tools bringing open science into the scientists' workflow, I believe I am qualified for the position of Project manager for the OpenAIRE-Advance project. I would be very motivated to participate in your efforts to link the different scientific outputs under a unique platform, and I would be honoured to follow in Dr Ross-Hellauer footsteps and implement open science in the scientists' daily life.

During my scientific career (Phd in biology in 2006, 10 years post-doctoral research), I have gathered experience in team management and communication strategies, especially inside the "Forschungsgruppe Biogenic amines" (FOR 1363) a DFG supported effort involving collaboration between a dozen of labs, and during my time as the director of the AOCF, which involved a lot of coordination between the different preclinical researchers and our team. My abilities to rapidly analyse problematics, find creative solutions and implement them has been highly valued by my peers.  As a member of the march for science berlin team, I got to be literate on public relation principles and guidelines, and got the chance to experience a great team work that lead to a very successful output. As a co-organiser of the open con satellite in Berlin (2016 and 2017), I got experience in relative big workshop organisation.

I have been interested in open science and its advocacy for years. Both on a theoretical (see \url{https://f1000research.com/articles/6-1151/v2}) and a practical perspective: All my work since 2009 has been published gold open access, with data and code published alongside; A paper in revision also use protocols.io reference for methods and the author reagent table format for material (https://doi.org/10.6084/m9.figshare.5398600.v1). My current project is an open science project aiming at creating a repository for mice home cage monitoring data using OSF, zenodo and github (\url{https://github.com/jcolomb/HCS_analysis}). I did present open science principles in two Drosophila neurobiology specific conference in 2004 and 2006, where I was selected for oral presentations twice (see \url{https://www.slideshare.net/JulienColomb/p-02-2014}). 

My interest in computer and programming (I am a self-taught R programmer) made me particularly aware of the poor data management  habits of scientific labs. With my experience I can also see the time waisted because of the absence of data management plans. My experience as the founder of Drososhare (a web-platform to facilitate lab material peer to peer transactions) introduced me to IT solutions development concepts like the importance of user experience, marketing channels, agile development and key performance indicators. I believe this experience would be very valuable at OpenAIRE, since "tools are only great if the are used". 

In brief, the person getting this position will have a great chance to make a difference in the open science landscape and I would be delighted to be that person. I am convinced I would be an asset for OpenAIRE and I would welcome any chance to convince you of my suitedness for the role. 

Dr. Julien Colomb


\vspace {1.cm} 

Referees:

Prof. York Winter, Berlin, york.winter@charite.de
 
 
Prof. Bjoern Brembs, Regensburg, bjoern@brembs.net
