\section* {Motivation letter}

I hereby apply for the position "Leitung fuer das GO FAIR Unterstuetzungs- und Koordinierungsbuero" inside the programs GO FAIR
\vspace {0.5cm} 
  % \emph{A crapy (software) tool that is used is better than a perfect one without user.}

Dear reader,

If you are looking for a creative scientist with IT knowledge, someone with a very strong background in (open) science, reproducible research and critical thinking, someone who is passionate about open FAIR data, someone who can work in a team to promote open science practices and policies, we should definitively meet.

I am an experienced neuro-geneticist with a decade of experience in complex behavioural data production and analysis in fundamental research, a field where one learns quickly that data quality and experiment speed are correlated positively. I am the first author of the first ever published paper with a living figure \url{https://f1000research.com/articles/3-176/v2}, and I have published all the data and code of my papers since 2009 (under CC0 licenses, mostly on Figshare, publications appeared of course in gold open access journals). My latest project is an open science project where I am developing a repository for mice home cage monitoring data. Independently, I have been involved in project aiming at teaching data management and open science practices (\url{opensciencemooc.eu}, \url{https://www.youtube.com/watch?v=M-Zod8o7nTg}, \url{http://access2perspectives.com/julien-colomb/}). I was also very interested in ontologies, open linked data and semantic web solutions, while I did not had the possibility to participate in this effort directly, yet.

 I am also proud to be part of thee march for science Berlin 2017 organisation team, which is still active developing other political relevant actions; as well as being co-leading the open science berlin meetup group.
I  have experience in team and project management (I could put the concepts learned in the "Moderation \& management" course during my time directing the animal outcome core facility), and have been very interested in the science of science during the last year, as indicated by my participation to discussions about peer review that ended in a paper published recently \url{https://f1000research.com/articles/6-1151/v2}. 
 I am also the happy father of two lovely children (4 and 1.5 years).

%\newpage 

I am  a quick learner, stress resistant and I like to work on ambitious projects lead by small teams. My current contract, a 50\% position at the HU, ends in March 2018. 
Since I left the lab in 2014, I have been trying to find new paths for my career and,
 I have been involved in many different projects in parallel. I have learned to manage my time and the team work more effectively and I am currently looking for a financially stable way to focus my energy on one single project. 

In brief, I feel like if we would work together, we could make a difference in the open science landscape,
and I am looking forward to meet you in person.

Please reach to me for additional information.

Dr. Julien Colomb


\vspace {1.cm} 

Referees:

Prof. York Winter, Berlin, york.winter@charite.de
 
 
Prof. Bjoern Brembs, Regensburg, bjoern@brembs.net
