\section* {Motivation letter}

I hereby apply for the position wissenschaftliche/r Mitarbeiter/in f\"{u}r die Kontaktstelle Forschungsdatenmanagement am Michael Stifel Center Jena. (92/2018)

\vspace {0.5cm} 
  % \emph{A crapy (software) tool that is used is better than a perfect one without user.}

Dear M. Gerlach,

I am a neuro-biologist whose interest in the optimisation of the scientific process has brought to promote open science [1,2] and I am passionate about open FAIR data and reproducible research (using R related tools). I have been working as an enthusiastic and pragmatic science manager, and advocating that efficacy and efficiency goes hand in hand with being more open and transparent.
%
I have written and managed 2 personal grants (2009-2012) while one DFG grant was obtained and managed in collaboration with Prof. Brembs (2011-2014). As scientific manager of the animal outcome core facility at the Charit\'{e}, I have enjoyed working at the service of other scientists, helping them to design and perform their experiment (as well as manage their data and data analysis).

Recently, I have been developing a 2 days workshop on data literacy and management, taking a pragmatic approach to these questions [3], and mostly emphasizing that data management is a way to save  time for scientists. I have been particularly interested in your approach combining consulting and training services, as it appeared to be a very effective approach in Switzerland. 
%
 I would be thrilled to take the challenge to develop tools to evaluate the efficiency of the work, while aiming at such a large spectrum of research topics. In addition, I would be very interested to learn more about the practice of "data driven science" at your contact. 

Owing to my education, my skill set, my goal-oriented and practical thinking, as well as my zeal, I believe I am qualified for the position and can help you reach your objectives in time.
Since I left the lab in 2014, I have been trying to find new paths for my career and I have been involved in many different projects in parallel. I have learned to manage my time and the team work more effectively and I am currently looking for a financially stable way to work on open science. 

 I am looking forward to meet you in person, and would be happy to answer any request for further information.


Dr. Julien Colomb



%\vspace {.5 cm} 


\newpage



%\rubrique {references}


%
Referees:
\begin{itemize}[noitemsep,topsep=0pt]
\item Prof. York Winter, Berlin, york.winter@charite.de
 
\item Prof. Bjoern Brembs, Regensburg, bjoern@brembs.net
\end{itemize}

References
\begin{itemize}[noitemsep,topsep=0pt]

\item [1] \url{https://www.meetup.com/Berlin-Open-Science-Meetup/}
\item [2]  \url{https://www.slideshare.net/JulienColomb/p-02-2014})
%\item [2] \url{https://opensciencemooc.eu/}
\item [3] \url{http://www.rpubs.com/j_colomb/256819}


\end{itemize}
\newpage 

