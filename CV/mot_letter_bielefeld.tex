\section* {Motivation letter}

I hereby apply for the \textsl{•}research position in the EU project OpenAIRE-Advance (wiss18012)
\vspace {0.1cm} 
  % \emph{A crapy (software) tool that is used is better than a perfect one without user.}

Dear Dr. Schirrwagen,

I am a former neuro-biologist whose interest in the optimisation of the scientific process has brought to promote open science and open data. I have worked as an enthusiastic and pragmatic science manager, who advocates that efficacy and efficiency goes hand in hand with being more open and sharing data in interoperable formats. I am also a autodidact R Shiny developer and founded a company running an internet platform in 2012 (drososhare.net). I have therefore experience in directing small and larger software development, with the user experience as a main concern.

Owing to my education, my skill set, my goal-oriented and practical thinking, as well as my zeal, I believe I am qualified for the position. There are many different tools emerging in the field of research data management and I am very curious to see the development of the next generation repositories.

I have been running multiple projects in parallel for the last years (Drososhare, scientific projects, meetups organisation, teaching open science, open source programming, ...) and I have learned ways to keep up with all the organisational tasks and the different communication channels.
 My abilities to rapidly analyse problematics, find creative solutions and implement them has been highly valued by my peers, while my talents in successful meeting organisation and  efficient project management, was an asset for the team. 
 %


I have been advocating for open science for years, both on theoretical (see \url{https://f1000research.com/articles/6-1151/v2}) and  practical levels (living figure, publication of data, code and computer readable material and method section, open source software). My current project is an open science project creating a distributed repository for mice home cage monitoring data using OSF, zenodo and github (\url{https://github.com/jcolomb/HCS_analysis}). I did advocate for open science principles in two Drosophila neurobiology specific conference in 2004 and 2006, where I was selected for oral presentations twice (see \url{https://www.slideshare.net/JulienColomb/p-02-2014}).


I believe the institutional, grassroots and commercial initiatives shall try to coordinate and modulate their efforts to develop the science and science communication of the future. Accordingly, I have have been pushing the development of open science communities (open science meetups series, mozilla science lab), networking in the startup scene for (open) science (labfolder, digital science, knowledge unlatched) and learning about the activities at the institutional level (OpenAIRE, FOSTER, FORCE11, TIB). I hope you will give me the opportunity to achieve a serious rise in the amount and quality of open science and shared research data while working at the  Bielefeld University Library.


Dr. Julien Colomb


\vspace {1.cm} 

Referees:
 \\
Prof. York Winter, Berlin, york.winter@charite.de
 \\
Prof. Bjoern Brembs, Regensburg, bjoern@brembs.net
