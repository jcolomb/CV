\section* {Motivation letter, IT Specialist Translational Science
Systems}

I hereby apply for the position "IT Specialist Translational Science
Systems" at Bayer, Berlin.
\vspace {0.5cm} 
  % \emph{A crapy (software) tool that is used is better than a perfect one without user.}

Dear recruiter,

If you are looking for a creative scientist with IT knowledge, someone with a very strong background in Biology and critical thinking, someone who can understand the need of life science researchers, depict how an IT solution can help them perform their tasks efficiently and effectively, as well as either find an existing tool that respond to their needs, or create that tool (either alone for very small projects, or in collaboration with a development team for larger ones), we should definitively meet.

I am an experienced neuro-geneticist with a decade of experience in complex behavioural data production and analysis in fundamental research, a field where one learns quickly that data quality and experiment speed are correlated positively.
More recently, I accumulated experience in the production and implementation of new data analysis standards in fundamental and preclinical research, mainly using R, Rmarkdown and Rshiny. I was also very interested in ontologies and semantic web solutions, while I did not had the possibility to participate in this effort directly, yet. In the last 2 years, I have been working with preclinical researchers, whose results I learned to critic fairly but systematically.

Independently, I have been running Drososhare GmbH and its IT solution development. Drososhare is a web-platform aiming at facilitating peer to peer transactions of GMOs in the life sciences. During the last four years, I learned the difficulties in communicating one's wish to IT developers and how to be more effective at it. I got involved and interested in the science and Berlin startup communities, learning about agile development, business models and specificities of the science market. I have been following the development of new tools like labfolder, authorea, overleaf, figshare, mendeley, paperhive, and protocols.io, most of them from their launch. I am also a fan of the grant4app program of Bayer and was present at different events organised by Jesus and his team.

I am one of the few who have both a very strong background in life science research (>10 years, 7 first-authored paper, 1 last-authored paper to come), and a decent background in creating IT solution to help the scientific process. I am also a quick learner, stress resistant and I like to work in a good atmosphere. In addition, I am currently looking for a position with new perspectives.

I am looking forward to meet you in person,

Dr. Julien Colomb


\vspace {0.5cm} 

Referees:

Prof. Bjoern Brembs, bjoern@brembs.net

Prof. York Winter, york.winter@charite.de
