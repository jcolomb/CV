\section* {Motivation letter}

I hereby apply for the position Project Manager for the European OpenAIRE-Advance project'' at Open-AIRE
\vspace {0.5cm} 
  % \emph{A crapy (software) tool that is used is better than a perfect one without user.}

Dear Open-AIRE team,

If you are looking for a creative scientist with IT knowledge, someone with a very strong background and interest in open science, reproducible research and critical thinking, someone who is passionate about open FAIR data, someone who can work in a team to promote open science practices and policies, we should definitively meet.

There is a strong political will to make open science happen and we are living an interesting time. The open access battle seem to be won and one is starting to looking beyond it toward reproducible research, open data, open materials and protocols and citizen science. However, there is still much to do on the front of publishing, with pre-prints and publishers monopolies problematics, to state only the obvious.  

As a strong open data advocate, I am very thrilled about the development of open science practices and workflows and am already working on teaching these novel habits  (\url{opensciencemooc.eu},  \url{http://access2perspectives.com/julien-colomb/}). My strong background in fundamental science (neuro-genetics) and data analysis has been strengthened by my interest in open science. I have published all the data and code of my papers since 2009 (all gold open access journals).

 I am also proud to be part of thee march for science Berlin 2017 organisation team, which is still active developing other political relevant actions; as well as being co-leading the open science berlin meetup group.
I  have experience in team and project management (I could put the concepts learned in the "Moderation \& management" course during my time directing the animal outcome core facility).
 I am also the happy father of two lovely children (4 and 1.5 years).

%\newpage 

I am  a quick learner, stress resistant and I like to work on ambitious projects lead by small teams. My current contract, a 50\% position at the HU, ends in March 2018. 
Since I left the lab in 2014, I have been trying to find new paths for my career and,
 I have been involved in many different projects in parallel. I have learned to manage my time and the team work more effectively and I am currently looking for a financially stable way to focus my energy on one single project. 

In brief, I feel like if we would work together, we could make a difference in the open science landscape,
and I am looking forward to meet you in person.

Please reach to me for additional information.

Dr. Julien Colomb


\vspace {1.cm} 

Referees:

Prof. York Winter, Berlin, york.winter@charite.de
 
 
Prof. Bjoern Brembs, Regensburg, bjoern@brembs.net
