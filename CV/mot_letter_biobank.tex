\section* {Motivation letter}

I hereby apply for the position "Koordinator/in fuer die IT-Vernetzung und Datenmanagement in Biobanken" at the german biobank node.
\vspace {0.5cm} 
  % \emph{A crapy (software) tool that is used is better than a perfect one without user.}

Dear Dr. Rufenach,

If you are looking for a creative scientist with IT knowledge, someone with a very strong background in Biology, reproducible research and critical thinking, someone who can understand the need of life science researchers, depict how an IT solution can help them perform their tasks efficiently and effectively, as well as either find or create a tool that respond to their needs, we should definitively meet.

I am an experienced neuro-geneticist with a decade of experience in complex behavioural data production and analysis in fundamental research, a field where one learns quickly that data quality and experiment speed are correlated positively. I am a strong advocate for open FAIR data.
More recently, I accumulated experience in the production and implementation of new data analysis standards in fundamental and preclinical research, mainly using R, Rmarkdown and Rshiny. I was also very interested in ontologies and semantic web solutions, while I did not had the possibility to participate in this effort directly, yet. In the last 2 years, I have been working with preclinical researchers, whose results I learned to critic fairly but systematically.

Independently, I have been running Drososhare GmbH and its IT solution development. Drososhare is a web-platform aiming at facilitating peer to peer transactions of GMOs in the life sciences. During the last four years, I learned the difficulties in communicating one's wish to IT developers and how to be more effective at it. I got involved and interested in the science and Berlin startup communities, learning about agile development, business models and specificities of the science market.

I am not so familiar with biobanks, since I used living animals in my research. But from what I have read, the problems in material identifications looks very similar, independently on the nature of the material (antibody, living animals, cell lines or tissue samples), and I would love to hear more about your strategy to minimise/solve these issues.  I am one of the few who have both a very strong background in life science research (>10 years, 7 first-authored paper, 1 last-authored paper to come), a decent background in creating IT solutions to help the scientific process and experience in team and project management (I could put the concepts learned in the "Moderation \& management" course during my time directing the animal outcome core facility). I am also a quick learner, stress resistant and I like to work in a good atmosphere. My current contract is a 50\% position at the HU, it ends in February 2018. 

I am looking forward to meet you in person,
please reach to me for additional information.

Dr. Julien Colomb


\vspace {0.5cm} 

Referees:

Prof. York Winter, york.winter@charite.de
