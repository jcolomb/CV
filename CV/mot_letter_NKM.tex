\section* {Motivation letter}

I hereby apply for the positions of research assistant, Kennziffer 31/2017, 

I used to work in fundamental research (neuro-genetics). In 2013, the boss of my lab moved to south Germany and let me in charge of the ongoing project, which involved one PhD student and one TA. At the end of the project in 2014, I left the academic race and worked on different projects at the HU, (projects involving an interaction with students and engineers, as well as a practical course organisation). I gathered more experience in scientific project management, and science communication.

During these last 3 years, my interest in open science grew steadily. I launched a "meetup group" (i.e. a group of interested people meeting about once a month) two years ago. The discussion I had during these meetings, brought me to think that the most effective strategy to foster open science in the scientific community is to teach the non-digital and digital skills needed to early career scientists. I am therefore preparing such a course and collaborating on the creation of a MOOC (massive online open course) for open science.

Most recently, I engaged myself in the organisation of the march for science in Berlin. Due to time constrains, my involvement was mostly restricted to providing ideas, helping with the logistics and social media as well as bringing positive thinking. I had the chance to meet great science communicator in the team, as well as discuss the challenges faced by our societies in the communication of scientific results and the scientific literacy of the citizens.

I have therefore a large knowledge of the open science and science communication topics, while I have been showing my capacity and interest to create pragmatic solutions to these problems. This position would  allow me to dedicate myself to the ongoing projects on open science and science communication, while giving me the opportunity to be in closer contact with citizen science projects. Although I have not followed a social science curriculum, I think my knowledge of the scientific process and the open science movement will be sufficient to cope with the scientific challenges inherent to this position. 

I am looking forward to meet you in person,
Dr. Julien Colomb

\vspace {1cm} 

Links:

Teaching open science: \url{https://reproducibleresearch.wordpress.com/}

Open science meetup: \url{http://www.meetup.com/Berlin-Open-Science-Meetup/}

March for science, orga team: \url{http://marchforscienceberlin.de/wer-wir-sind}

Scientific achievements: \url{http://orcid.org/0000-0002-3127-5520}

\vspace {0.5cm} 

Referees:

Prof. Bjoern Brembs, bjoern@brembs.net

