\begin{letter}{Helmoltz-zentrum Potsdam \\ Deutsches Geofosrschungszentrum\\ Potsdam}


%
\textbf {I hereby apply for the position Projektkoordinator/in im Bereich Open Access, inside the Deep Green project, Kennziffer 0774}
%\vspace {0.5cm} 
  % \emph{A crapy (software) tool that is used is better than a perfect one without user.}

\opening{Dear M. Pampel,}


I am a former neurobiologist who has been following open science and open access movements over the last five years. As a scientists, I have a long record of project presentations and collaborative working, while I have been gathering more experience in project and team management while working at a core facility (AOCF, Charit\'{e}) and in my current work at the research data management helpdesk in Jena. In order to gather more experience in team management and open access, I am  currently moderating an international open source project (www.reagents.io), I am part of the steering committee of the open science MOOC,  and I recently volunteered to help FORCE11 with their online communication strategies.
%





The green road to open access has a large potential in freeing research outputs from publishers copyright, while avoiding the great dangers of the gold route. I would be delighted to bring my enthusiasm and work on that topic. In particular, I would be interested to know about the relation between the deep green project and the FOSTER openminted project, as well as unpaywall and other similar initiatives.

During the last years, I have been meeting different actors of the open science community by participating to  conferences (open science barcamp, Force17, .csv conference,..) and by being invited to different interview for positions linked to open science and data management (TUHH, TIB, Th\"{u}ringer Hochschulen, dariah, labfolder, german biobank node,  Go FAIR, ...).
 I realized that I have still a lot to learn about the world of libraries and about the  most effective communication strategies to use in that community, and I would be highly motivated to learn while working in your team.
 
 I therefore believe that owing to my education, my skill set and my interest for the dissemination of open science practices, I am qualified for the position of Project coordinator at the Helmholtz center in Potsdam. I think it would be a great opportunity for me to learn to work with librarian. The announced flexibility in work-time and workplace is very attractive, as I have two children. I hope to get the pleasure to meet you personally for an interview.
 
 Sincerely yours,

Dr. Julien Colomb\\
Schillerpromenade 4\\
12049 Berlin
julien.colomb@fu-berlin.de

